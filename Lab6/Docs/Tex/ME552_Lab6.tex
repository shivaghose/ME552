%You can leave alone everything before Line 79.
\documentclass{article}
\usepackage{url,amsfonts, amsmath, amssymb, amsthm,color, enumerate, verbatim}
% Page layout
\setlength{\textheight}{8.75in}
\setlength{\columnsep}{2.0pc}
\setlength{\textwidth}{6.5in}
\setlength{\topmargin}{0in}
\setlength{\headheight}{0.0in}
\setlength{\headsep}{0.0in}
\setlength{\oddsidemargin}{0in}
\setlength{\evensidemargin}{0in}
\setlength{\parindent}{1pc}
\newcommand{\shortbar}{\begin{center}\rule{5ex}{0.1pt}\end{center}}
%\renewcommand{\baselinestretch}{1.1}
% Macros for course info
\newcommand{\courseNumber}{ME 552}
\newcommand{\courseTitle}{Mechatronics}
\newcommand{\semester}{Fall 2012}
\newcommand{\xxx}[1]{\textcolor{red}{#1}}
% Theorem-like structures are numbered within SECTION units
\theoremstyle{plain}
\newtheorem{theorem}{Theorem}[section]
\newtheorem{lemma}[theorem]{Lemma}
\newtheorem{corollary}[theorem]{Corollary}
\newtheorem{proposition}[theorem]{Proposition}
\newtheorem{statement}[theorem]{Statement}
\newtheorem{conjecture}[theorem]{Conjecture}
\newtheorem{fact}{Fact}
%definition style
\theoremstyle{definition}
\newtheorem{definition}[theorem]{Definition}
\newtheorem{example}{Example}
\newtheorem{problem}[theorem]{Problem}
\newtheorem{exercise}{Exercise}
\newtheorem{algorithm}{Algorithm}
%remark style
\theoremstyle{remark}
\newtheorem{remark}[theorem]{Remark}
\newtheorem{reduction}[theorem]{Reduction}
%\newtheorem{question}[theorem]{Question}
\newtheorem{question}{Question}
%\newtheorem{claim}[theorem]{Claim}
%
% Proof-making commands and environments
\newcommand{\beginproof}{\medskip\noindent{\bf Proof.~}}
\newcommand{\beginproofof}[1]{\medskip\noindent{\bf Proof of #1.~}}
\newcommand{\finishproof}{\hspace{0.2ex}\rule{1ex}{1ex}}
\def\therefore{\boldsymbol{\text{ }
\leavevmode
\lower0.4ex\hbox{$\cdot$}
\kern-.5em\raise0.7ex\hbox{$\cdot$}
\kern-0.55em\lower0.4ex\hbox{$\cdot$}
\thinspace\text{ }}}

\newenvironment{solution}[1]{\medskip\noindent{\bf Problem #1.~}}{\shortbar}

%====header======
\newcommand{\solutions}[4]{
%\renewcommand{\thetheorem}{{#2}.\arabic{theorem}}
\vspace{-2ex}
\begin{center}
{\small  \courseNumber, \courseTitle
\hfill {\Large \bf {#1} }\\
\semester, University of Michigan, Ann Arbor \hfill
{\em Date: #3}}\\
\vspace{-1ex}
\hrulefill\\
\vspace{4ex}
{\LARGE Lab Assignment #2}\\
\vspace{2ex}
\end{center}
\begin{trivlist}
\item \textsc{Team members:\\} {#4}
\end{trivlist}
\noindent
\shortbar
\vspace{3ex}
}
% math macros
\newcommand{\defeq}{\stackrel{\textrm{def}}{=}}
\newcommand{\Prob}{\textrm{Prob}}
\newcommand{\Lagr}{\mathcal{L}}
\newcommand{\Sens}{\mathcal{S}}
%==
\usepackage{graphicx}
\usepackage{xfrac}
\usepackage{amsmath}
\providecommand{\e}[1]{\ensuremath{\times 10^{#1}}}
\begin{document}
%%%%%%%%%%%%%%%%%%%%%%%%%%%%%%%%%%%%%%%%%%%%%%%%%
%\solutions{Your name}{Problem Set Number}{Date of preparation}{Collaborators}{Prover}{Verifiers}
\solutions{}{6: Inertial Sensors}{\today}{Shiva Ghose, @gshiva\\ John Peterson, @jrpeters\\ Peter Turpel, @pturpel\\ Chan-Rong Lin, @pmelin}
%%%%%%%%%%%%%%%%%%%%%%%%%%%%%%%%%%%%%%%%%%%%%%%%%
%\renewcommand{\theproblem}{\arabic{problem}} 
%%%%%%%%%%%%%%%%%%%%%%%%%%%%%%%%%%%%%%%%%%%%%%%%%
%
% Begin the solution for each problem by
% \begin{solution}{Problem Number} and ends it with \end{solution}
%
% the solution for Problem 
\section*{Teamwork Participation Pledge :: Team 1}

I attest that I have made a fair and equitable contribution to this lab and submitted 
assignment. \\

My signature also indicates that I have followed the University of Michigan Honor Code, 
while working on this lab and assignment.\\

I accept my responsibility to look after all of the equipment assigned to me and my team, 
and that I have read and understood the X50 Lab Rules.\\

\begin{table}[h]
\begin{center}
    \begin{tabular}{|c|c|c|}
        \hline
        \textbf{Name} & \textbf{Email}     & \textbf{ \ \ \ \ \  \ \  \ \ \ \ \  \ \ Signature  \ \ \ \ \  \ \ \ \ \ \ \  \ \ } \\ \hline
        	~& ~& ~\\
	~& ~& ~\\
	Shiva Ghose   & gshiva@umich.edu   & ~                  \\
	~& ~& ~\\
	~& ~& ~\\ \hline 
	~& ~& ~\\
	~& ~& ~\\
        John Peterson & jrpeters@umich.edu & ~                  \\ 
	~& ~& ~\\
	~& ~& ~\\ \hline 
	~& ~& ~\\
	~& ~& ~\\
        Peter Turpel   & pturpel@umich.edu & ~                  \\
	~& ~& ~\\
	~& ~& ~\\ \hline 
	~& ~& ~\\
	~& ~& ~\\
        Chan-Rong Lin   & pmelin@umich.edu & ~                  \\
	~& ~& ~\\
	~& ~& ~\\ \hline 
        \hline
    \end{tabular}
\end{center}
\end{table}

\newpage

\section{Accelerometer Wiring \& Configuration} 

\subsection{Configuration}

Figure \ref{wiring} shows the wiring diagram of the inertial sensor system. The accelerometer is mounted to a breadboard fixed to the the shaft clamp on the encoder such that the plane of the sensor PCB is perpendicular to the plane of axis of rotation of the encoder.  This is necessary because of the limited number of axes available on the rate gyro.  For this assignment we will be making use of the Y and Z axes of the accelerometer and the X axis of the rate gyro (standard and X4.5). We also connected the auto-zero and temperature sensor pins. We are not using the X axis of the accelerometer, the Y axis of the rate gyro (standard and Y4.5), the self-test, or the VRef pins. We used an Arduino Uno as a voltage regulator to provide the encoder with 5V and the IMU with 3.3V.m the Arduino itself was powered from a wall outlet via an AC-to-DC converter.   \\

\begin{figure}[hbt]
\begin{center}
\includegraphics[width = 16cm]{WiringDiagram.png}
\caption{Wiring Connections of Inertial Sensor System.}
\label{wiring}
\end{center}
\end{figure}

\xxx{Confirm that the wiring diagram is correct specifically whether we are using the high or low sensitivity outputs.}

\xxx{Still missing gyro details}

\xxx{We may want a picture of each axis including gyro}

\clearpage

\subsection{Pin Functionality}

\subsubsection{Accelerometer}
Figures \ref{accelPins} and \ref{accelFunc} show the arrangement of the pins on the accelerometer and its functional block diagram, respectively. \\

\begin{figure}[hbt]
\begin{center}
\includegraphics[width = 7cm]{ADXL335Pins.png}
\caption{Pin Configuration of the ADXL335 3-Axis Accelerometer.}
\label{accelPins}
\end{center}
\end{figure}

\begin{figure}[hbt]
\begin{center}
\includegraphics[width = 11cm]{ADXL335Functional.png}
\caption{Functional Diagram of the ADXL335 3-Axis Accelerometer.}
\label{accelFunc}
\end{center}
\end{figure}

\textbf{Pins 1, 4, 9, 11, 13, and 16 - NC}\\
These pins are not connected to the accelerometer's internal circuitry. They are not functional, but they can be connected to ground (the COM pins).\\

\textbf{ Pin 2 - ST}\\
This pin is for the self-test feature of the accelerometer. When the pin is set high by applying the supply voltage an electrostatic force is applied to the accelerometer axes. This results in a known change in the voltage output from each axis. If the change is not seen in the output, this is an indication that there is a problem with the accelerometer. The pin can be connected to ground or left unconnected when not needed. THe PCB provides a connection for this pin, which we did not use.\\

\textbf{Pins 3, 5, 6, and 7 - COM}\\
These pins connect various elements of the accelerometer chip to ground. The PCB provides a connection for ground from the power supply and internally connects this to all of the COM pins. We connected this to the ground terminal of the Arduino.\\

\textbf{Pins 8, 10, and 12 - Z$_{out}$, Y$_{out}$, and X$_{out}$ (respectively)}\\ 
Analog voltage output of the 3 axes of the accelerometer. The accelerometer outputs voltage corresponding to acceleration in the each direction with a range of $\pm$ 3g according to equation \ref{ParamID_EQ1}. The Z axis accelerometer nominally has greater noise density and lower bandwidth compared to the X and Y axes, which are nominally equal. The PCB provides connections for these pins. We connected pins 8 and 10 to the NI breakout board to read with LabView, but did not use pin 12 because that axis (X) was oriented along the shaft's axis of rotation.\\

\textbf{Pins 14 and 15 - V$_s$}\\
Supply voltage for the accelerometer. The accelerometer has a functional supply range of 1.8 V to 3.6 V, with a maximum range of -0.3 V to 3.6 V (based on operation reasonably safe from damage, not functional operation). The accelerometer is ratiometric with typical performance values given based on a V$_s$ of 3 V. Because the output is ratiometric, it is important to use a regulated power supply providing constant voltage. The PCB provides a connection for these pins, which we connected to the 3.3 V terminal of an Arduino Uno.\\ 

\subsubsection{Rate Gyro}
Figures \ref{gyroPins} and \ref{gyroFunc} show the arrangement of the pins on the rate gyro and its functional block diagram, respectively. \\

\begin{figure}[hbt]
\begin{center}
\includegraphics[width = 8cm]{IDG500Pins.png}
\caption{Pin Configuration of the IDG500 Dual-Axis Gyro.}
\label{gyroPins}
\end{center}
\end{figure}

\begin{figure}[hbt]
\begin{center}
\includegraphics[width = 16cm]{IDG500Functional.png}
\caption{Functional Diagram of the IDG500 Dual-Axis Gyro.}
\label{gyroFunc}
\end{center}
\end{figure}

\textbf{Pins 1 and 20 - X-OUT and Y-OUT (respectively)}\\
Analog output voltage of the X and Y axes of the gyro. The gyro outputs voltage proportional to the angular rate about the X and Y axes. The outputs are not ratiometric and have a nominal full scale range of $\pm$500$^{\circ}$ and sensitivity of 2 $\frac{mV}{^{\circ}/s}$. A low-pass filter is provided internally. The PCB provides connections to these pins. We connected X-OUT to the NI breakout board to read with LabView, but did not connect Y-OUT because our setup did not rotate about that axis.\\

\textbf{Pins 2, 8, 26, 27, and 28 - GND}\\
These pins connect various elements of the gyro chip to ground. The PCB provides a connection for ground from the power supply and internally connects this to all of the GND pins. We connected this to the ground terminal of the Arduino.\\

\textbf{Pins 3, 4, 17, and 18 - NC}\\
These pins are not connected to the gyro's internal circuitry. The data sheet notes that they can be used for PCB trace routing.\\

\textbf{Pins 5 and 16 - X4.5IN and Y4.5IN (respectively)}\\
These pins are inputs to amplifiers which can be connected to X-OUT and Y-OUT to provide higher sensitivity outputs. The connections are optional but are provided internally by the PCB.\\

\textbf{Pins 6 and 15 - XAGC and YAGC (respectively)}\\
These pins provide connections for amplitude control capacitors. 0.22 $\mu$F capacitors are internally used to connect these pins to ground. The capacitors are part of a control loop which controls the amplitude of the internal gains to ensure the sensitivity does not vary with temperature.\\

\textbf{Pins 7 and 14 - X4.5OUT and Y4.5OUT (respectively)}\\
Analog output voltage of the X and Y axes of the gyro. These pins increase the output of X-OUT and Y-OUT by a gain of 4.5 - effectively increasing the sensitivity at the expense of range. The output has a nominal full scale range of $\pm$110$^{\circ}$ and sensitivity of 9.1 $\frac{mV}{^{\circ}/s}$. The PCB provides connections to these pins. We connected X4.5OUT to the NI breakout board to read with LabView, but did not connect Y4.5OUT because our setup did not rotate about that axis.\\

\textbf{Pins 9 and 19 - VDD}\\
Supply voltage for the gyro. The gyro has a functional supply range of 2.7 V to 3.3 V, with a maximum range of -0.3 V to 6.0 V (based on operation reasonably safe from damage, not functional operation). The gyro is not ratiometric and all performance values are based on a VDD of 3 V. The PCB provides a connection for these pins, which we connected to the 3.3 V terminal of an Arduino Uno. A voltage regulator (likely a MIC5205) is included on the PCB betweent the VDD pins and the 3.3 V input connection to ensure the gyro receives 3.0 V. \\ 

\textbf{Pins 10, 11, 13, 21, and 25 - RESV}\\
These pins are reserved and are not connected.\\

\textbf{Pin 12 - CPOUT}\\
Provides an internal connection from ground to a charge pump regulator via a 0.1 $\mu$F capacitor. A charge pump is used for DC/DC regulation and can boost voltages without needing an inductor. \xxx{Might need more explanation here}\\

\textbf{Pin 22 - VREF}\\
A precision reference output which is nominally 1.35 V. Ideally VREF will be a constant 1.35 V and is used with the auto-zero function to reset the zero-rate output to its nominal value of 1.35 V in the event of steady-state drift. The PCB provides a connection to this pin which we did not use.\\

\textbf{Pin23 - PTATS}\\
Provides an output voltage from a temperature sensor within the gyro chip. Output is proportional to temperature with a nominal bias voltage of 1.25 V and a sensitivity of 4 mV/$^{\circ}$C. The PCB provides a connection to this pin which we connected to the NI breakout board and read in LabView (though we did not end up using the data).\\

\textbf{Pin 24 - AZ}\\
This pin is an input to an auto-zero funciton for the X and Y axis gyros. When activated, the auto-zero resets the zero-rate output to be equal to VREF. The auto-zero is typically used when the gyro is not moving to prevent steady-state drift and is supposed to increase the usable range of the high sensitivity outputs (X4.5OUT and Y4.5OUT). The auto-zero is intiated by setting the pin high with a pulse between 2 and 1500 $\mu$s. The PCB provides a connection to this pin with we connected to a digital output of the NI breakout board. In our LabView VI we boolean momentary switch to this output so the user could activate the auto-zero, but this was rarely used.\\ 

\section{Modeling Assumptions}
% there is probably more stuff
\begin{itemize}
\item{Assume that outputs of each axis of the accelerometer and the gyro are independent of one another}
\item{Assume zero bias of each accelerometer axis is stationary} % we may eliminate this assumption if we do online calibration
\item{$g = 9.81 \, \left( \sfrac{m^2}{s}\right) $}
\end{itemize}

\section{Parameter Identification}

\subsection*{Accelerometers}
The basic behavior of an accelerometer, sensitive in the $\hat{u}$ direction is given by equation \ref{ParamID_EQ1} for quasi-static accelerations.  Where $V_{out}$ is the output voltage of the sensor, $V_{bias}$ is the output of the sensor under no acceleration, $V_{supply}$ is the supply voltage to the sensor, and $\Sens$ is a linear approximation of sensor behavior.

\begin{equation}
V_{out} = V_{bias} + \Sens \ddot{u} \quad V_{bias} \approx \frac{V_{supply}}{2}
\label{ParamID_EQ1}
\end{equation}

This response to acceleration can be also be used to measure the acceleration due to gravity along the sensitive axis of an accelerometer allowing the angle between the accelerometer and vertical to be determined.  Let $\theta$ be the measured angle from vertical to the $\hat{y}$ direction of our system.  We can consider two situations, when the sensitive axis is aligned with $\hat{y}$ axis and when it is perpendicular to it.  Let $Y$ and $Z$ denote these situations respectively which correspond to the $\hat{y}$ and $\hat{z}$ axes of our accelerometer.  Then the output voltage of each axis is given as follows:

\begin{equation}
V_{Y} = V_{Ybias} + \Sens_{Y} g \cos(\theta) \quad V_{Z} = V_{Zbias} + \Sens_{Z} g \sin(\theta)
\label{ParamID_EQ2}
\end{equation}

\begin{figure}
\begin{center}
\includegraphics[width = 11cm]{Accelerometer_Cartoon.png}
\caption{Y and Z axis accelerometer}
\label{Accel_cartoon}
\end{center}
\end{figure}

We measured the bias voltage of each axis by fixing the accelerometer such that the axis was perpendicular to the force of gravity and recording the voltage output.  We obtained the bias voltage by averaging these voltages, and we also obtained a measure of noise along each axis by computing the variance of the obtained voltages.  We then conducted a simple experiment where the accelerometer was rotated at slow speeds throughout a complete circle while both axis voltage outputs and encoder angle measurements were recorded.  The equations in \ref{ParamID_EQ2} are linear in the sensitivity values, so we obtained each value through simple least squares regression.  These results are shown in table \ref{ParamID_T} \xxx{not the right table?} \\


$$sigma = rms \, Noise = Noise \, Density \sqrt{BW * 1.6} $$

\begin{table}
\begin{center}
	\begin{tabular}{|c|c|}
		\hline
		Bandwidth $(Hz)$ & 50 \\ 
		Noise Density $Y_{out} \, \sfrac{\mu g}{\sqrt{Hz}} \, rms$ & 150 \\ 
		Noise Density $Z_{out} \, \left(\sfrac{\mu g}{\sqrt{Hz}} \, rms\right)$ & 300 \\ 
		Sensitivity $\left( \sfrac{mV}{g} \right)$ & 300 \\  
		$\sigma^2 \,Y_{out} \, (V^2)$ & $1.62 \e{-7}$ \\ 
		$\sigma^2 \,Z_{out} \, (V^2)$ & $6.48 \e{-7}$ \\[0.1cm] \hline
		
	\end{tabular}
\label{ParamID_AccelT}
\end{center}
\end{table}

% adjust the data sheet values to correspond to if we had a supply voltage of 3.295 Volts, it is ratio metric so it should work
% values as reported by matlab SY = 0.034002931081677, SZ = 0.033903141625302, EY = 2.642681255078259\e{-5}, EZ = 2.774354019095888\e{-4}
\begin{table}
\begin{center}
    \begin{tabular}{|c|c|c|c|c|}
        \hline
        Axis                              & Y Specification & Y   Fit & Z Specification                & Z    Fit                \\ \hline
        $V_{bias} \, (V)$                & 1.5        & 1.617433958  & 1.5         & 1.671451851           \\ 
        $\sigma^2 \, (V^2)$                  & $1.62 \e{-7}$      & 1.24553\e{-5}    & $6.48 \e{-7}$       & 0.001852504           \\ 
        $\Sens \, \left(\sfrac{V s^2}{m} \right)$                & 0.03058104           & $0.03400293$  & 0.03058104   & $0.03390314$     \\ 
        Fitting Error $(V)$  & - & $2.642681\e{-5}$  & - & $2.774354\e{-4}$ \\
        \hline
    \end{tabular}
\label{ParamID_T}
\caption{Fitting error denotes average error per sample.  Note that the bias voltage and sensitivities quoted in the data sheet are for a supply voltage of 3.0 V, while we are using a supply voltage of 3.295 V}
\end{center}
\end{table}

We see that as reported in the data sheet, the Z-axis is noisier than the Y-axis, but it is to a much greater degree than the data sheet would suggest.  We also see that this extra noise is reflected in the higher fitting errors for the Z-axis data.  

\begin{figure}
\begin{center}
\includegraphics[width = 13cm]{YaxisAccel_Calib.png}
\label{YaccelCalib}
\caption{Expected vs Measured Voltages for the Y axis accelerometer}
\end{center}
\end{figure}

\begin{figure}
\begin{center}
\includegraphics[width = 13cm]{ZaxisAccel_Calib.png}
\label{ZaccelCalib}
\caption{Expected vs Measured Voltages for the Z axis accelerometer}
\end{center}
\end{figure}

\subsection*{Rate Gyro}

% note that these calibration results were obtained from the natural frequency with the normal pendulum

A rate gyro gives outputs a voltage proportional to the angular rate about the sensitive axis of the gyro with some offset, the Zero Rate which is not ratiometric.  

$$V_{out} = ZeroRate + \Sens_{Gyro} \dot{\theta} $$

However even of the time period of a few seconds, we encountered several difficulties in trying to apply such a model.  First of all, the rate gyro's response is limited to higher frequencies requiring us to use an oscillating calibration run at the natural frequency of our pendulum sensor set up, of approximately 1.41 Hz.  The data was trimmed to only include the region of the response of a high enough frequency to be within the bandwidth of the sensor.  Secondly, even over this shorter calibration run, the Zero Rate was no constant.  To overcome these difficulties, we fit a slightly more complex model to obtain a good value for the sensitivity.

$$V_{out} = A t + B + \Sens_{Gyro} \dot{\theta} $$

Both $A$ and $B$ vary with time and other variables, such as temperature, so $B$, the zero rate of the gyro is determined on line and A is neglected, so their values for this particular run are irrelevant.  

Assuming that the bandwidth of the DAQ and sensor system is 140 $(Hz)$ as described in the data sheet.  Noise density was computed from the total RMS noise value in the data sheet using the following formula, which was also then used to compute the expected noise over the lower bandwidth.

$$sigma = rms \, Noise = Noise \, Density \sqrt{BW * 1.6} $$

\begin{table}
\begin{center}
	\begin{tabular}{|c|c|}
		\hline
		
		$\Sens_{Gyro} \, \left( \sfrac{V s}{rad} \right)$ & 0.36 \\ 
		Total RMS Noise $(mV \, rms)$ & 0.8 \\ 
		Noise Density $\left( \sfrac{V}{\sqrt{Hz}}\right)$ & 2.0 \e{-5} \\ 
		Bandwidth $(Hz)$ & 140 \\ 
		$\sigma \, \left( V^2 \right)$ & 8.96 \e{-8} \\ 
		\hline
	\end{tabular}
\label{ParamID_DatasheetGyro}
\caption{Performance parameters for the IDG-500 obtained from the data sheet.  \emph{Note that Noise Density and $\sigma^2$ were derived from data sheet values.}}
\end{center}
\end{table}

% quoted sensitivity = 1 / 8.635380
\begin{table}
\begin{center}
    \begin{tabular}{|c|c|c|c|c|}
        \hline
        ~   & Natural Frequency 1 & Natural Frequency 2 & Manual 1 & Manual 2\\ \hline
	Data Set Frequency (Hz)  & 1.41  & 1.76 & 2.42 & 5.13\\
        $\Sens_{Gyro} \, \left( \sfrac{V s}{rad} \right)$  & 0.115803  & 0.11672 & 0.11358 & 0.114821\\
	$\sigma \, \left( V^2 \right)$  & $3.773 \e{-6}$ & $4.1077\e{-6}$ & 4.7680 \e{-6} & 4.1615\e{-5}\\
	Fitting Error $(rad)$ &  $3.358 \e{-4}$ & $4.206 \e{-4}$ & 2.814 \e{-4} & 2.050 \e{-4}\\
        \hline
    \end{tabular}
\label{ParamID_TGyro}
\caption{Gyro Sensitivity at a variety of frequencies within the full scale range.}  
\end{center}
\end{table}





\begin{figure}
\begin{center}
\includegraphics[width = 13cm]{rateGyroCalibResultsS8_636380.png}
\label{gyroCalib}
\caption{Actual Angle vs Fitted Angle from integrating rate gyro output for Natural Frequency 1 trial}
\end{center}
\end{figure}

\clearpage
\section{Low Frequency Characterization}

%%%%%%%%%%%%%%%%%%%%%%%%%%%%%%%%%%%%%%%%%%%%%%%%%%%%%%%%%
\subsection{Horizontal Accelerometer}

\subsubsection{Angle Measurement}

Using a single accelerometer in the horizontal configuration is equivalent to just using the Z axis in our configuration.  Where $V_{Z}$ is given as follows: 

$$ V_{Z}(\theta) = V_{Zbias} + \Sens_{Z} g \sin(\theta) $$

$\theta$ can be easily computed by the following expression:

\begin{equation}
\theta = \sin^{-1}\left( \frac{V_{Z} - V_{Zbias}}{\Sens_{Z} g}\right) 
\label{horizontalEQ}
\end{equation}

When using a single accelerometer in the horizontal configuration aliasing occurs for $|\theta| \geq \sfrac{\pi}{2}$.  It is important to note that the arcsine function has a restricted domain and is only valid on inputs between -1 and 1.  Noise and inaccurate estimates of the bias voltage, sensitivity and gravity can all drive the input quantity beyond this range.  To address this, we simply saturate inputs to restrict them to the valid range.

$$\theta = \left\{ 
	\begin{array}{l l}
		\sin^{-1} \left(\frac{V_{Z} - V_{Zbias}}{\Sens_{Z} g} \right) & {\bf{if }} \, \left| \frac{V_{Z} - V_{Zbias}}{\Sens_{Z} g} \right| \leq 1 \\[6pt]
		\frac{\pi}{2} & {\bf{if }} \, \frac{V_{Z} - V_{Zbias}}{\Sens_{Z} g} > 1 \\[6pt]
		-\frac{\pi}{2} & {\bf{if} } \, \frac{V_{Z} - V_{Zbias}}{\Sens_{Z} g} < -1
	\end{array} \right. $$

%%%%%%
% Need real Data for part i
\xxx{still need the real results here?}
%%%%%%

\subsubsection{Sensitivity}

The relationship between sensor output voltage and angle is non-linear, but we are able to linearize about a particular operating point, for this experiment, $\theta = 0$, and obtain an estimate of the sensitivity.

$$ V_{Z}(\theta) \approx V_{Zbias} + \Sens_{Z} g \sin(0) + \Sens_{Z} g \cos(0) \left(\theta - 0\right) = V_{Zbias} + \Sens_{Z} g \theta $$

For the horizontal accelerometer, we expect the sensitivity to be $\Sens_{Z} g$. 

\subsubsection{Noise}

An estimate for the expected noise in measurement of $\theta$ can be obtained by performing a variance projection through the partial derivative of equation \ref{horizontalEQ}.

$$ \frac{\partial \theta}{\partial V_{Z}} =  \frac{1}{\sqrt{(\Sens_{Z} g)^2 - (V_{Z} - V_{Zbias})^2}}$$

$$ \sigma^2_{\theta} = \frac{\partial \theta}{\partial V_{Z}} \sigma^2_{V_{Z}} \frac{\partial \theta}{\partial V_{Z}} $$

$$ \sigma^2_{\theta} = \frac{\sigma^2_{V_{Z}}}{(\Sens_{Z} g)^2 - (V_{Z} - V_{Zbias})^2}$$

For $\theta \approx 0$ we would expect the noise of our angle measurements to be given by:

$$ \theta \approx 0 \quad \sigma^2_{\theta} \approx \frac{\sigma^2_{V_{Z}}}{(\Sens_{Z} g)^2}$$

The following data was collected for a fixed vertical pendulum, $\alpha = 0$.  It seems that there was a discrepancy betweeen the earlier recorded bias voltage for the Z-axis, and the bias voltage for this trial.  Besides that difference, the mean and variance very closely resemble the expected values, but as noted earlier the noise is significantly worse than promised by the data sheet.  

% comparison to real data here
\begin{table}
\begin{center}
    \begin{tabular}{|c|c|c|}
        \hline
        ~                   & Theoretical  & Actual \\ \hline
        $\sigma^2_{V_{Z}} \, (V^2)$    & 0.001852504            &  0.00185231      \\ 
        $\sigma^2_{\theta} \, (rad^2)$ & 0.01674716            & 0.0169008      \\ 
        $\mu_{V_{Z}} \, (V)$       & 1.671451851            &  1.701057      \\ 
        $\mu_{\theta} \, (rad)$      & 0            &  -1.219912 \e{-5}     \\
        \hline
    \end{tabular}
\label{Noise_horizontal_T}
\caption{Comparison of theoretical noise in $\theta$ with measured noise in $\theta$ for the horizontal accelerometer alone. \emph{Note that the theoretical variance of voltage and bias were measured in another trial.}}
\end{center}
\end{table}

%%%%%%%%%%%%%%%%%%%%%%%%%%%%%%%%%%%%%%%%%%%%%%%%%%%%%%%%%
\subsection{Vertical Accelerometer}

\subsubsection{Angle Measurement}

Using a single accelerometer in the vertical configuration is equivalent to using just the Y axis in our configuration.  Where $V_{Y}$ is given as follows:

$$ V_{Y} = V_{Ybias} + \Sens_{Y} g \cos(\theta) $$

Then $\theta$ is given by the following expression:

\begin{equation}
\theta = \cos^{-1}\left( \frac{V_{Y} - V_{Ybias}}{\Sens_{Y} g}\right)
\label{verticalEQ}
\end{equation}

When using a single accelerometer in this configuration aliasing occurs for angles beyond the range $0 \leq \theta \leq \pi$.  This range is much less useful than the range for a single accelerometer in the horizontal configuration because we are unable to read negative angles immediately next to our starting condition, $\theta = 0$.  The arccosine function also has a limited domain, from 1 to -1, and we simply limit it to the valid range as in the horizontal accelerometer case above.

$$\theta = \left\{ 
	\begin{array}{l l}
		\cos^{-1}\left( \frac{V_{Y} - V_{Ybias}}{\Sens_{Y} g}\right) & {\bf{if }} \, \left| \frac{V_{Y} - V_{Ybias}}{\Sens_{Y} g} \right| \leq 1 \\[6pt]
		0 & {\bf{if }} \, \frac{V_{Y} - V_{Ybias}}{\Sens_{Y} g} > 1 \\[6pt]
		-\pi & {\bf{if} } \, \frac{V_{Y} - V_{Ybias}}{\Sens_{Y} g} < -1
	\end{array} \right. $$

%%%%%%%
% Real Results
\xxx{still need the real results here}
%%%%%%%

\subsubsection{Sensitivity}

However linearizing about $\theta = 0$ for the accelerometer in the vertical configuration is not particularly successful,  yielding a relationship that does not depend on $\theta$ at all.  This result is expected from our earlier assessment of the angle ambiguity between positive values of $\theta$ and negative values.  

$$V_{Y}(\theta) \approx V_{Ybias} + \Sens_{Y} g \cos(0) - \Sens_{Y} g \sin(0) (\theta - 0) = V_{Ybias} + \Sens_{Y} g $$

\subsubsection{Noise}

As before, the following noise trial was conducted for a fixed vertical pendulum, with $\alpha = 0$. We can again use variance projection to estimate the noise in our angle measurements, taking the partial derivative of equation \ref{verticalEQ}.

$$ \frac{\partial \theta}{\partial V_{Y}} = -\frac{1}{\sqrt{(\Sens_{Y} g)^2 - (V_{Y} - V_{Ybias})^2}}$$

$$ \sigma^2_{\theta} = \frac{\partial \theta}{\partial V_{Y}} \sigma^2_{V_{Y}} \frac{\partial \theta}{\partial V_{Y}} $$

$$ \sigma^2_{\theta} = \frac{\sigma^2_{V_{Y}}}{(\Sens_{Y} g)^2 - (V_{Y} - V_{Ybias})^2}$$

As we would expect, this appears identical to the equation for noise of the horizontally mounted accelerometer.  The difference in behavior is simply the voltage around which we linearize.  Substituting the value for $V_{Y}$ about $\theta = 0$  yields a surprising result.

% is this result just a quirk of the linearization?

$$ \theta \approx 0 \quad \sigma^2_{\theta} \approx \frac{\sigma^2_{V_{Y}}}{(\Sens_{Y} g)^2 - (V_{Ybias} + \Sens_{Y} g - V_{Ybias})^2} = \infty$$

What has happened, is that the above calculation neglected the effects of our earlier signal conditioning.  While the expected voltage is $V_{Ybias} + \Sens_Y g$.  The expected value of the quantity within the arccosine is not 1.  Because of the saturation conditioning used to prevent arccosine from failing, the actual distribution is no longer Gaussian, and most importantly is not symmetric about 0 which leads to the relatively large deviation from the expected mean of desired mean of 0.  Instead the expectation is given by:

$$ E \left[ \frac{V_{Y} - V_{Ybias}}{\Sens_{Y} g} \right] \approx 0.5 \cdot 1 + 0.5 \cdot \big(1 - \sigma_{V_Y} \sqrt{\sfrac{2}{\pi}} \big) = 0.99859205 $$

$$ E\left[ \cos^{-1} \left(  E \left[ \frac{V_{Y} - V_{Ybias}}{\Sens_{Y} g} \right]  \right)  \right] = 0.0530713 \, (rad) $$

The variance of this sort of distribution is given by: 

% think I'm a bit off in this calculation, I didn't quite take into account the fact that it isn't just a half gaussian, we've got that spike at 1, http://en.wikipedia.org/wiki/Half-normal_distribution
% what I have is a clipped normal distribution
\xxx{verify this one calculation, it seems a little bit off - - - - - - -   V}
$$ \sigma_{arg1}^2 = \frac{\sigma^2_{V_Y}}{(\Sens_Y g)^2} = 1.11939534 \e{-4} \quad \sigma_{arg2}^2 = \sigma_{arg}^2 \left(1 - \frac{2}{\pi} \right) = 4.06766136 \e{-5}$$

Then all that is left to do is to linearize the arccosine about this new mean:

$$ \frac{d}{dx} \cos^-1(0.99859205) = -\frac{1}{\sqrt{1 - 0.99859205^2}} = -18.85143028 $$

\emph{Because the function is highly nonlinear in this region, the variance estimate is not going to particularly accurate.}  

Finally the variance can be obtained:

$$ \sigma_{\theta}^2 = \left( \frac{d}{dx} \cos^-1(0.99859205) \right) ^2 \sigma_{arg2}^2 = 0.01445551 $$

% comparison to real data here
\begin{table}
\begin{center}
    \begin{tabular}{|c|c|c|}
        \hline
        ~                   & Theoretical  & Actual \\ \hline
        $\sigma^2_{V_{Y}} \, (V^2)$    & 1.24553\e{-5}  & 5.807739 \e{-5}      \\ 
	$\sigma^2_{\theta} \, (rad^2)$ & 0.01445551           &  0.009754764     \\ 
	$\mu_{V_{Y}} \, (V)$       & 1.95100270            & 1.95268     \\
        $\mu_{\theta} \, (rad)$      & 0.0530713            & 0.09371065      \\
        \hline
    \end{tabular}
\label{Noise_vertical_T}
\caption{Comparison of theoretical noise in $\theta$ with measured noise in $\theta$ for the Y axis accelerometer alone. \emph{Note that the theoretical variance of voltage and expectation was calculated from sensitivity and bias voltage measured in another trial.}}
\end{center}
\end{table}

The trial verifies our suspicions that the mean would not be 0, and in fact in the trial, the variance was greater, than in our measurements causing the mean angle to be further from 0 than expected. 

%%%%%%%%%%%%%%%%%%%%%%%%%%%%%%%%%%%%%%%%%%%%%%%%%%%%%%%%%
\subsection{Two Accelerometers}

\subsubsection{Angle Measurement}

We can overcome the aliasing issue  presented in the two configuration by using both the horizontal and vertical, the Z and Y axis, simultaneously.  

$$ \cos(\theta) = \frac{V_Y-V_{Ybias}}{\Sens_{Y} g} \quad \sin(\theta) = \frac{V_{Z} - V_{Zbias}}{\Sens_{Z} g} $$

Combining the two equations gives the following:

$$ \tan(\theta) = \frac{\sin(\theta)}{\cos(\theta)} = \left(\frac{V_{Z} - V_{Zbias}}{\Sens_{Z} g}\right) \left( \frac{\Sens_{Y} g}{V_Y-V_{Ybias}} \right) = \frac{\Sens_{Y}}{\Sens_{Z}} \left( \frac{V_{Z} - V_{Zbias}}{V_{Y} - V_{Ybias}} \right)$$

Using the atan2 function avoids the quadrant ambiguities present in the ordinary $\tan^{-1}$ function giving us an expression for $\theta$ valid for all angles.

$$\theta = \text{atan2}\big( \Sens_{Y} \left( V_{Z} - V_{Zbias}\right),  \Sens_{Z} \left( V_{Y} - V_{Ybias}\right) \big)$$

\xxx{Signal Conditioning?}

%%%%%%%
% Still need real results
\xxx{Real Results here}
%%%%%%%

\subsubsection{Sensitivity}

Sensitivity of our angle measurement, $\theta$ is simply given by the partial derivatives, shown in equations \ref{dualAccel_partialVY} and \ref{dualAccel_partialVZ} of our expression for $\theta$ shown in equation \ref{dualAccel_EQ}.

\begin{equation}
\theta = \tan^{-1} \left( \frac{\Sens_{Y} \left( V_{Z} - V_{Zbias}\right)}{\Sens_{Z} \left( V_{Y} - V_{Ybias}\right)} \right) = f(V_{Y},V_{Z})
\label{dualAccel_EQ}
\end{equation}

\begin{equation}
\frac{\partial f}{\partial V_{Y}} = \frac{\Sens_Z \Sens_Y \left(V_{Z} - V_{Zbias} \right)}{\Sens^2_Y \left(V_Z - V_{Zbias} \right) ^2 + \Sens^2_Z \left( V_Y - V_{Ybias}\right)^2}
\label{dualAccel_partialVY}
\end{equation}

\begin{equation}
\frac{\partial f }{\partial V_Z} = \frac{\Sens_Z \Sens_Y \left(V_{Y} - V_{Ybias} \right)}{\Sens^2_Y \left(V_Z - V_{Zbias} \right) ^2 + \Sens^2_Z \left( V_Y - V_{Ybias}\right)^2}
\label{dualAccel_partialVZ}
\end{equation}

For $\theta \approx 0$, the sensitivity to the Y axis drops to 0 and the sensitivity to the Z axis is given by the same expression as before. 

\xxx{do they want sensitivity from V to $\theta$ or the other way around}

$$ \frac{\partial f}{\partial V_{Y}} \approx 0  \quad \frac{\partial f }{\partial V_Z} = \frac{1}{\Sens_{Z} g}$$

\subsubsection{Noise}

Determining the expected noise is a bit more complicated for functions of several variables and requires a bit more math, shown below.   

Let 

$$ \vec{x} = \left[ V_Y, V_Z \right]^T $$

Then

$$ \theta = f(\vec{x}) \approx J|_{\vec{x}_0} \left( \vec{x} - \vec{x}_{0}\right) \quad J = \left[ \frac{\partial f}{\partial V_{Y}}, \frac{\partial f }{\partial V_Z} \right] $$

$$ \sigma^2_{\theta} = \Sigma_{\theta} = J \Sigma_{\vec{x}} J^T  \quad 
\Sigma_{\vec{x}} = \left[
\begin{matrix}
\sigma^2_{V_{Y}}  & 0 \\
0 & \sigma^2_{V_{Z}} 
\end{matrix} \right]$$

$$ \sigma^2_{\theta} = \left(\frac{\partial f}{\partial V_{Y}}\right)^2 \sigma^2_{V_{Y}} + \left(\frac{\partial f }{\partial V_Z} \right)^2 \sigma^2_{V_{Z}} $$

For our trial conducted with the pendulum in the vertical position, $\alpha = 0$, we can simply plug in $\theta = 0$ into our earlier equations to estimate the noise in this angle measurement.  For $\theta = 0$, the noise is approximately equal to the noise using only the horizontal axis of the accelerometer, which turns out ot be almost exactly true, though the variance does not quite match the actual value considering just the horizontal axis.

$$ \theta \approx 0 \quad V_{Z}(\theta) \approx V_{Zbias} + \Sens_{Z} g \theta \quad V_{Y}(\theta)  \approx V_{Ybias} + \Sens_{Y} g $$

$$ \frac{\partial f}{\partial V_{Y}} \approx 0  \quad \frac{\partial f }{\partial V_Z} = \frac{1}{\Sens_{Z} g}$$

$$ \sigma^2_{\theta} \approx \left(0\right)^2 \sigma^2_{V_{Y}} + \left(\frac{1}{\Sens_{Z} g} \right)^2 \sigma^2_{V_{Z}} = \left( \frac{\sigma_{V_Z}}{\Sens_{Z} g} \right)^2$$

% comparison to real system
\begin{table}
\begin{center}
    \begin{tabular}{|c|c|c|}
        \hline
        ~                   & Theoretical  & Actual \\ \hline
        $\sigma^2_{V_{Z}} \, (V^2)$    & 0.001852504            & 0.001852312     \\ 
	$\mu_{V_{Z}} \, (V)$       & 1.671451851            & 1.7010571      \\ 
	$\sigma^2_{V_{Y}} \, (V^2)$ & 1.24553\e{-5}		& 5.8077389 \e{-5} \\
	$\mu_{V_{Y}} \, (V)$       & 1.95100270            & 1.619099      \\ 
        $\sigma^2_{\theta} \, (rad^2)$ & 0.01674716             &  0.01647518     \\ 
        $\mu_{\theta} \, (rad)$      & 0            & -0.002390427      \\
        \hline
    \end{tabular}
\label{Noise_dual_T}
\caption{Comparison of expected noise and actual noise for $\theta = 0$ using both accelerometer axes.}
\end{center}
\end{table}


%%%%%%%%%%%%%%%%%%%%%%%%%%%%%%%%%%%%%%%%%%%%%%%%%%%%%%%%%
\subsection{Rate Gyro}

\subsubsection{Angle Measurement}

As mentioned earlier, the output voltage of the rate gyro is assumed to be linearly related to the the angular velocity about the sensitive axis of the gyro as shown in the equation repeated below.  The Zero Rate and the sensitivity of the gyro are assumed to be time invariant.  Solving for $\dot{\theta}$ yields:

$$V_{Gyro}(t) = ZeroRate_{Gyro} + \Sens_{Gyro} \dot{\theta}(t) $$

$$\dot{\theta}(t) = \frac{V_{Gyro}(t) - ZeroRate_{Gyro}}{\Sens_{Gyro}} $$

Integrating the expression for $\dot{\theta}$ yields $\theta$, and this integral is approximated by applying the trapezoid rule for numerical quadrature.

$$ \theta(t) = \int_0^t \dot{\theta}(\tau) d\tau = \int_0^t \frac{V_{Gyro}(\tau) - ZeroRate_{Gyro}}{\Sens_{Gyro}} d\tau$$
$$ \theta(t) \approx \sum_{i=1}^n \left(\frac{t(i) - t(i-1)}{ \Sens_{Gyro}} \right) \left( \frac{V_{Gyro}(i) + V_{Gyro}(i - 1)}{2} - ZeroRate_{Gyro} \right) $$

%%%%%
% results here
\xxx{need results here}
%%%%%

\xxx{Signal Conditioning}

\subsubsection{Sensitivity}

As described above, there is no linear relationship between the voltage reported by the rate gyro and the angular position of the pendulum.  

\subsubsection{Noise}

Because $\theta$ is given as a function of the sum of previous measurements, errors in $\theta$ are cumulative and will diverge from the true value over time.  This is reflected in time increasing variance, here expressed in terms of the number of measurements made.  Note that the following derivation neglects the variance in the zero rate which will only worsen performance and assumes that gyro variance is independent of time.

$$ \sigma_{\theta}(t) \approx \left( \frac{\sigma_{Gyro}^2}{\Sens_{Gyro}} \right) \left\{ \frac{t(1) - t(0)}{2} + \sum_{i = 1}^{n-1} \big( t(i) - t(i-1) \big) + \frac{t(n) - t(n-1)}{2} \right\} \approx \left( \frac{\sigma_{Gyro}^2}{\Sens_{Gyro}} \right) t$$

\emph{Where n is the sample taken at time, t.} \\

It is important to note, that despite this high variance, the actual output will look quite smooth because of the high degree of correlation between subsequent position estimates.  \\

The following trial was conducted for a fixed vertical pendulum angle, $\alpha = 0$.  Because our noise trial was done stationary, the gyro's lack of sensitivity to low frequency motion was not important, and its superior noise characteristics to the other sensors lead to very little noise in $\theta$.  In this trial, the Zero Rate was estimated from data within the first second.  Notice the slight discrepancy between the mean gyro voltage, and the zero rate.  This discrepancy leads to the divergent behavior seen in figure \ref{static_Gyro}.   Compared to the effect of zero rate drift, the effect of accumulating errors due to integration is essentially negligible.  

\begin{table}
\begin{center}
    \begin{tabular}{|c|c|}
        \hline
        ~                     & Actual \\ \hline
        $\sigma^2_{V_{Gyro}} \, (V^2)$              &  1.0179178 \e{-6}      \\ 
        $\sigma^2_{\theta} \, (rad^2)$             & 3.3233613 \e{-6}      \\ 
        $\mu_{V_{Gyro}} \, (V)$                &  1.360523703     \\ 
	$ ZeroRate_{Gyro} \, (V)$ 	     &  1.360478444 \\
        $\mu_{\theta} \, (rad)$               &  0.002894257     \\
        \hline
    \end{tabular}
\label{Noise_horizontal_T}
\caption{Observed Rate Gyro noise and resulting angle estimate for fixed $\theta = 0$.}
\end{center}
\end{table}

\begin{figure}
\begin{center}
\includegraphics[width = 12cm]{rateGyro_Static.png}
\label{static_Gyro}
\caption{Rate Gyro Angle estimate with static pendulum at $\theta = 0$}
\end{center}
\end{figure}

\clearpage
%%%%%%%%%%%%%%%%%%%

\section{Filter Results}

\clearpage
%%%%%%%%%%%%%%%%%%%

\section{Low Frequency Characterization}

\clearpage
%%%%%%%%%%%%%%%%%%%

\section{Higher Frequency Characterization}

To determine the range of motiion that could be measured, the pendulum was moved through a full rotation at highspeed. To determine the magnitude, phase, and other performance measures as a function of frequency, the pendulum was allowed to osciallate naturally with the shaft collars at differect positions along its length to alter the center-of-mass. Then, the pendulm was manually oscilated at several frequencies (as consistently as possible). In all cases the low-pass filter time constant and the kalman filter variance were kept constant (0.05 and 0.07, respectively) for all of the accelerometer calculations. The time constant of the high-pass filter used with the rate gyro's was changed with every trial.\\  

\subsection{Horizontal Accelerometer}

Figure \ref{full_horizontal} shows the angular range of the horizontal accelerometer when the pendulum undergoes a full rotation at high speed. The acclerometer was capable of an ouput range of $\pm$90$^{\circ}$ and is capable of distinguishing direction of rotation, but in this case the encoder was outputting 0 to 2$\pi$, leading to  some disagreement. Had the encoder been outputting -$\pi$ to $\pi$, there would be more agreement within the accelerometer's range

\begin{figure}[hbt]
\begin{center}
\includegraphics[width = 12cm]{FullRotation_Horizontal.png}
\label{full_horizontal}
\caption{Horizontal accelerometer with high-speed, full rotation of pendulum}
\end{center}
\end{figure}

\begin{figure}[hbt]
\begin{center}
\includegraphics[width = 12cm]{NormalMass_Horizontal.png}
\label{normal_vertical}
\caption{Horizontal accelerometer at pendulum's natural frequency}
\end{center}
\end{figure}

\subsubsection{High Frequency}

\subsubsection{High Frequency with Horizontal Arm Disturbance}

\clearpage
%%%%%%%%%%%%%%%%%%%%%

\subsection{Vertical Accelerometer}
\subsubsection{High Frequency}

\begin{figure}[hbt]
\begin{center}
\includegraphics[width = 12cm]{FullRotation_Vertical.png}
\label{full_vertical}
\caption{Vertical accelerometer with high-speed, full rotation of pendulum}
\end{center}
\end{figure}

\begin{figure}[hbt]
\begin{center}
\includegraphics[width = 12cm]{NormalMass_Vertical.png}
\label{normal_vertical}
\caption{Vertical accelerometer at pendulum's natural frequency}
\end{center}
\end{figure}

\subsubsection{High Frequency with Horizontal Arm Disturbance}
\clearpage
%%%%%%%%%%%%%%%%%%%%%
\subsection{Two Accelerometer}
\subsubsection{High Frequency}

\begin{figure}[hbt]
\begin{center}
\includegraphics[width = 12cm]{FullRotation_Dual.png}
\label{full_Dual}
\caption{Dual accelerometers with high-speed, full rotation of pendulum}
\end{center}
\end{figure}

\begin{figure}[hbt]
\begin{center}
\includegraphics[width = 12cm]{NormalMass_Dual.png}
\label{normal_dual}
\caption{Dual accelerometers at pendulum's natural frequency}
\end{center}
\end{figure}

\subsubsection{High Frequency with Horizontal Arm Disturbance}

\clearpage
%%%%%%%%%%%%%%%%%%%%%
\subsection{Rate Gyro}
\subsubsection{High Frequency}

\begin{figure}[hbt]
\begin{center}
\includegraphics[width = 12cm]{FullRotation_Gyro.png}
\label{full_gyro}
\caption{Rate gyro with high-speed, full rotation of pendulum}
\end{center}
\end{figure}

\begin{figure}[hbt]
\begin{center}
\includegraphics[width = 12cm]{NormalMass_Gyro.png}
\label{normal_gyro}
\caption{Rate gyro at pendulum's natural frequency}
\end{center}
\end{figure}


\begin{figure}[hbt]
\begin{center}
\includegraphics[width = 12cm]{FullRotation_Gyro45.png}
\label{full_gyro45}
\caption{4.5 Rate gyro with high-speed, full rotation of pendulum}
\end{center}
\end{figure}

\begin{figure}[hbt]
\begin{center}
\includegraphics[width = 12cm]{NormalMass_Gyro45.png}
\label{normal_gyro45}
\caption{4.5 Rate gyro at pendulum's natural frequency}
\end{center}
\end{figure}

\subsubsection{High Frequency with Horizontal Arm Disturbance}

\clearpage
%%%%%%%%%%%%%%%%%%%%%
\section{Sensor Fusion}

\subsection{Construction}
We make use of the all three inertial sensors discussed in the previous sections, the horizontal and vertical accelerometers and the rate gyro.  Using both accelerometers eliminates sign ambiguities to provide angle measurements for a full 360 degrees, but it suffers from poor performance at higher angular rates because of dynamic effects.  To overcome this issue we use the rate-gryo which suffers from poor performance at lower angular rates, but is able to provide accurate angles at higher rates.  By combining these sensors we can achieve performance across a broader range of frequencies and angles than by using a single sensor alone.  

As in the set up that we have used for the entire lab, the Y-axis of the accelerometer points in the vertical direction at $\theta = 0$ while the Z-axis points in the horizontal direction, while the X-axis of the gyro is parallel to the axis of the encoder and perpendicular to the Y and Z axes of the accelerometer.  

\subsection{Filter Design}

We use a 1st order low pass filter applied to the angle, $\theta$ output by the the pair of accelerometer and a 1st order high pass filter applied to the rate, $\dot{\theta}$ output by the rate gyro.  Both filters have a cut off frequency of \xxx{FREQUENCY $\approx$ 1 Hz} \xxx{describe tuning} 

\subsection{Block Diagram}

For encoder angle $\alpha$, we have already shown that the pair of accelerometers provide an estimate $\theta$ and the rate gyro provides $\dot{\theta}$.  Applying a low pass filter to $\theta$ and a high pass filter to $\dot{\theta}$ yields the following two expressions which can be added to obtain $\theta$ which is approximately equal to $\alpha$.  

$$ \theta \frac{1}{\tau s + 1} = \frac{\theta}{\tau s + 1} $$
$$ \dot{\theta} \frac{\tau}{\tau s + 1} = \theta s \frac{\tau}{\tau s + 1} = \theta \frac{\tau s}{\tau s + 1} $$
$$ \theta \frac{1}{\tau s + 1} + \dot{\theta} \frac{\tau}{\tau s + 1} = \theta \left(\frac{1}{\tau s + 1} + \frac{\tau s}{\tau s + 1}  \right) = \theta \approx \alpha $$

\begin{figure}
\begin{center}
\includegraphics[width = 14cm]{fusion_BlockDiagram.png}
\label{fusion_Block}
\caption{Block Digram for Sensor Fusion}
\end{center}
\end{figure}

\subsection{Experimental Evaluation}

\end{document}